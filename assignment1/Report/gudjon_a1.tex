\documentclass[12pt]{article}
\usepackage{graphicx}
\usepackage{subcaption}
\usepackage{mwe}
\usepackage{amsmath}
\usepackage{listings}
\usepackage{color} %red, green, blue, yellow, cyan, magenta, black, white
\definecolor{mygreen}{RGB}{28,172,0} % color values Red, Green, Blue
\definecolor{mylilas}{RGB}{170,55,241}

%\usepackage{lingmacros}
%\usepackage{tree-dvips}
%\usepackage{blindtext}
%\usepackage[utf8]{inputenc}

\renewcommand{\thesubsection}{\thesection.\alph{subsection}}

\begin{document}

\lstset{language=Matlab,%
    %basicstyle=\color{red},
    breaklines=true,%
    morekeywords={matlab2tikz},
    keywordstyle=\color{blue},%
    morekeywords=[2]{1}, keywordstyle=[2]{\color{black}},
    identifierstyle=\color{black},%
    stringstyle=\color{mylilas},
    commentstyle=\color{mygreen},%
    showstringspaces=false,%without this there will be a symbol in the places where there is a space
    numbers=left,%
    numberstyle={\tiny \color{black}},% size of the numbers
    numbersep=9pt, % this defines how far the numbers are from the text
    emph=[1]{for,end,break},emphstyle=[1]\color{red}, %some words to emphasise
    %emph=[2]{word1,word2}, emphstyle=[2]{style},    
}


\title{Assignment 1}
\author{Gudjon Einar Magnusson}

\maketitle


\section{}


\section{}

\subsection{}
The solution that I found by hand is similar but not the same as the one returned by qr function in Matlab.
Two of the axis directions are flipped but it does form a orthonormal basis.

\subsection{}

To find the gram matrix I wrote the following function:
\lstinputlisting{findGram.m}


\subsection{}
To find a rank k version of matrix I wrote the following function.

\lstinputlisting{rankk.m}


For $A_{(2)}$ the two norm and the frobenius norm both return the same value.

\begin{center}
$\lvert\lvert A - A_{(2)} \rvert\rvert_{2} = 0.55921$

$\lvert\lvert A - A_{(2)} \rvert\rvert_{F} = 0.55921$
\end{center}


\end{document}